%%%%%%%%%%%%%%%%%%%%%%% file template.tex %%%%%%%%%%%%%%%%%%%%%%%%%
%
% This is a general template file for the LaTeX package SVJour3
% for Springer journals.          Springer Heidelberg 2010/09/16
%
% Copy it to a new file with a new name and use it as the basis
% for your article. Delete % signs as needed.
%
% This template includes a few options for different layouts and
% content for various journals. Please consult a previous issue of
% your journal as needed.
%
%%%%%%%%%%%%%%%%%%%%%%%%%%%%%%%%%%%%%%%%%%%%%%%%%%%%%%%%%%%%%%%%%%%
%
% First comes an example EPS file -- just ignore it and
% proceed on the \documentclass line
% your LaTeX will extract the file if required
\begin{filecontents*}{example.eps}
%!PS-Adobe-3.0 EPSF-3.0
%%BoundingBox: 19 19 221 221
%%CreationDate: Mon Sep 29 1997
%%Creator: programmed by hand (JK)
%%EndComments
gsave
newpath
  20 20 moveto
  20 220 lineto
  220 220 lineto
  220 20 lineto
closepath
2 setlinewidth
gsave
  .4 setgray fill
grestore
stroke
grestore
\end{filecontents*}
%
\RequirePackage{fix-cm}
%
%\documentclass{svjour3}                     % onecolumn (standard format)
%\documentclass[smallcondensed]{svjour3}     % onecolumn (ditto)
\documentclass[smallextended]{svjour3}       % onecolumn (second format)
%\documentclass[twocolumn]{svjour3}          % twocolumn
%
\smartqed  % flush right qed marks, e.g. at end of proof
%
\usepackage{graphicx}
%
% \usepackage{mathptmx}      % use Times fonts if available on your TeX system
%
% insert here the call for the packages your document requires
%\usepackage{latexsym}
% etc.
%
% please place your own definitions here and don't use \def but
% \newcommand{}{}
%
% Insert the name of "your journal" with
% \journalname{myjournal}
%
\begin{document}

\title{Interdisciplinary Water Reservoir Management%\thanks{Grants or other notes
%about the article that should go on the front page should be
%placed here. General acknowledgments should be placed at the end of the article.}
}
\subtitle{A holistic approach to reservoiur management and it's implications?\\ If so, write it here}

%\titlerunning{Short form of title}        % if too long for running head

\author{Daus, Milan       \and
       Koberger, Katharina\and Koca, Kaan, \and Beckers, Felix \and Encinas-Fernandes, Jorge \and Weisbrod, Barbara
}

%\authorrunning{Short form of author list} % if too long for running head

\institute{Daus, Milan \at
             Physical geography Albert-Ludwigs-University Freiburg, Schreiberstrasse 20, 79098 Freiburg (Germany) \\
              Tel.: +49 761 203 3515}\\
              \email {milan.daus@geographie.uni-freiburg.de}           
           \and
           {Koberger, Katharina \at
               Physical geography Albert-Ludwigs-University Freiburg. Schreiberstrasse 20, 79098 Freiburg (Germany)}\\
\email {katharina.koberger@geographie.uni-freiburg.de}
\and
{Koca, Kaan \at
Institute for Modelling Hydraulic and Environmental Systems. Pfaffenwaldring 61, 70569 Stuttgart (Germany)}\\
\email {kaan.koca@iws.uni-stuttgart.de}
\and
{Beckers, Felix \at
Institute for Modelling Hydraulic and Environmental Systems. Pfaffenwaldring 61, 70569 Stuttgart (Germany)}\\
\email {felix.beckers@iws.uni-stuttgart.de}
\and
{Encinas-Fernandes, Jorge \at
Limnological Institute of the University of Constance. Mainaustraße 252, 78464 Konstanz (Germany)}\\
\email {jorge.encinas@uni-konstanz.de}
\and
{Weisbrod, Barbara \at
Human and Environmental Toxicology. Universitätsstraße 10, 78464 Konstanz (Germany)}\\
\email {barbara.weisbrod@uni-konstanz.de}
\date{Received: date / Accepted: date}
% The correct dates will be entered by the editor


\maketitle

\begin{abstract}
Water resources, especially freshwater, are one of the most important resources for humankind. The available resources have been managed for centuries to ensure accessibility and continuous use of water over the entire year. Reservoirs are a common way to store and retain water enabling a multitude of uses like storage of drinking and irrigation water, recreation, flood protection, navigational requirements and hydro energy production. Today, few reservoirs serve only one purpose thereby requiring a management of demands and interests present. Since the projects will cause negative impacts alongside desired advantages both on a local and global scale, it is even more urgent to develop a common management framework that is able to incorporate and mitigate these interests and make them visible within a discoursive environment in order to avoid conflicts early on. The research interest on reservoirs is manifold, yet a comprehensive and integrative information tool is not available. Therefore, the tool presented here is based on the results from the CHARM (Challenges of Reservoir Management) project as well as the condensed outcome of relevant literature to aid and enhance knowledge in water reservoir management. The project focused on five different topics of interest in relation to existing water reservoirs, namely: sedimentation of reservoirs, biostabilization of fine sediments, toxic cyanobacteria (blooms), greenhouse gas emissions from reservoirs and social contestation respectively approval. These five research foci contributed to the topics and setup of an information tool which can be found alongside this publication.
\keywords{Reservoir Management \and Sediments \and Bioflm \and Cyanobacteria \and Greenhouse Gas Emissions \and Societal Implications}
% \PACS{PACS code1 \and PACS code2 \and more}
% \subclass{MSC code1 \and MSC code2 \and more}
\end{abstract}

\section{Introduction}
\label{intro}
Water and energy are two of the most important resources on our planet. Dams and reservoirs provide and store these resources based on the implementation of infrastructure to retain water. Due to population growth and changes in lifestyle the demands towards the watercourses are constantly rising and diversifying. This incorporates interests like drinking and irrigation water supply, flood protection, energy production or navigational requirements (Glaser, 2014; DPA, 2019). It is estimated that around 58.000 large dams are built, planned or under construction worldwide, a trend that will likely accelerate in the future (Mulligan et al. 2020). In Germany, dam construction with an emphasis on hydropower began around 1900 and reached its peak during the second half of the last century (Blackbourn and Rennert, 2008). At present, there are 371 reservoirs that meet the ICOLD (International Commission on Large Dams) definition of large dams (more than 15 m dam height or 3 million cubic meter storage capacity) (ICOLD, 2013). The construction of dams and reservoirs does not only radically change the area directly affected by the artificial lake itself, but the entire runoff system as well as the catchment area (Döll et al., 2009). There is change in the flow regime, sediment deposition in the reservoirs, accumulation of nutrients in the (stored) water body, cyanobacteria blooms and methane and carbon dioxide emissions from the water column to name but a few (Beckers et al., 2018). These issues in reservoir management were motivating the scientific project CHARM (CHallenges of Reservoir Management) from the Universities Stuttgart, Constance and Freiburg in Baden-Wuerttemberg (Germany) to research those challenges and positively contribute to future demands regarding managing (large) water reservoirs. 

\section{Sediments}
\label{sec:1}
The global construction of approx. 58,000 large dams (ICOLD, 2019; Mulligan et al., 2020) has considerably influenced delivery of sediment from land to oceans by fragmenting the river network (Nilsson et al., 2005) resulting in an estimated over 100 Gt sediment trapped in reservoirs (Syvitski et al., 2005). When a river enters a reservoir, sediments carried by the river settle due to decreasing flow velocities and turbulence levels. Consequently, sedimentation reduces reservoir storage that was designed for water and thus hinders the services of the reservoirs. Due to the binding properties of fine sediments (large surfaces compared to their volume) coupled with biostabilization (see Section 2), pollutants, nutrients and organic matter also accumulate in the reservoirs (e.g. Majerová et al., 2018; Pohlert et al., 2011), impairing water quality (see Section 3) and leading to emissions of CO2 and CH4 (see Section 4). Additionally, since incoming and outgoing sediment loads are different, hydrology, morphology and ecology of downstream channels are also negatively impacted (Collier et al., 2000; Graf, 2006; Magilligan and Nislow, 2005; Wu et al., 2019). Vörösmarty et al. (1997) and Vörösmarty et al. (2003) estimated that approximately 40% of the global river discharge was blocked by large reservoirs with a storage capacity greater than 0.5 km3, yielding an estimation of 4-5 Gt sediments trapped in reservoirs annually. Mahmood (1987), Yoon (1992) and Bruk (1996) estimate a reservoir storage reduction by 1% per year. Coupling this information with current storage of major dams (6863.5 km3) and assuming a dry bulk density of 1200 kg/m3, an annual storage loss is calculated to be ∼82 Gt/yr, which is approx. 20-fold larger than the estimation by Vörösmarty et al. (2003). It must, however, be noted that storage loss is not equally distributed across the world since the quantity of sediments deposited in the reservoirs depends strongly on physical (e.g., shape) and operational characteristics of the reservoir (e.g., pumping) as well as  the hydrological characteristics of the catchment (e.g., Haun et al., 2013). In addition, hydrological variability as a result of climate change (Haddeland et al., 2014) and human interactions may also increase sediment loads, leading to increased reservoir sedimentation. Therefore, development and implementation of sustainable sediment management techniques to maintain sufficient reservoir storage over the long term has become increasingly important in pursuit of meeting rising energy and water demands.
Traditionally, reservoirs are designed for a lifetime of 50 or 100 years (Annandale et al., 2016; Morris and Fan, 1998), with abandoning or decommissioning the reservoir in mind, leading to societal, environmental and economical problems beyond the design life. They are usually equipped with a dead storage volume to account for sedimentation. However, depending on the given boundaries (e.g., sediment yield, shape and volume of the reservoir), reservoir sedimentation may claim more reservoir volume than the designed dead storage volume and an active reservoir management becomes necessary. Indeed, based on the estimated approx. 69 km3 annual storage loss, 30% and 80% of the global reservoir capacity will be lost by 2050 and 2100, respectively if the reservoirs are managed ineffectively. Therefore, particularly in views of competing land uses as well as pushing social (see Section 5) and environmental issues, it is critical that sediment management strategies and facilities are planned considering long-term reservoir sedimentation already during the design and construction phase of the new reservoirs to ensure that the implemented measures are most effective and sustainable over the long-term period (>100 years). Furthermore, existing reservoirs must be adapted to sustainable management practices inasmuch as is possible (Annandale et al., 2016). Since the 1990s several guidelines have been released related to sediment management in reservoirs (e.g., Morris & Fan, 1998; Batuca & Jordaan, 2000). These documents provide specific guidelines to evaluate the feasibility of different sediment management strategies (Atkinson, 1996; Shen, 1999; White, 2001). As each reservoir is unique, the final decision on an appropriate management strategy must be taken with care. Sediment management strategies to reduce reservoir sedimentation and maintain reservoir volume can be listed under three different approaches (Kondolf et al., 2014):
-reducing the sediment yield originating from the catchment by controlling erosion upstream
-minimizing the sediment depositions in the reservoir by managing flows during periods of high flows
-recovering already lost reservoir volume by removing deposited sediment using various techniques.
\section{Biostabilisation of fine sediments}
\label{sec:1}
The predominant microbial lifestyle in aquatic ecosystems is characterised by multicellular and multispecies communities (Fischer and Pusch, 2001; Flemming and Wuertz, 2019; Geesey et al., 1978; Lozupone and Knight, 2007), flourishing between their self-synthesized three-dimensional matrix of extracellular polymeric substances (EPS) (Flemming and Wingender, 2010; Stoodley et al., 2002). These diverse microbial communities are capable of colonizing various solid-water interfaces, with sediment being excellent substrata (Battin et al., 2016; Gerbersdorf and Wieprecht, 2015) and possess common features that distinguish them as biofilm (Flemming, 2020; Flemming and Wingender, 2010; Gerbersdorf and Wieprecht, 2015). The transition of microorganisms from planktonic lifestyle to the biofilm lifestyle is controlled by a range of environmental conditions among which the local hydrodynamics are of paramount importance (Berke et al., 2008; Gerbersdorf and Wieprecht, 2015). Surface-attached microorganisms deactivate the expression of the genes involved in motility and activate the genes involved in adhesion and biofilm development (Tuson and Weibel, 2013). Surface association and subsequent biofilm formation provides these microorganisms with critical advantages. As opposed to a single planktonic lifestyle, enhanced availability of essential nutritional resources, increased interactions of microorganisms within the surface associated biofilm-matrix and distinct, mostly diverse compositions lead to higher metabolic activity. This results in high survival, wide tolerance to environmental conditions and higher reproduction potential to the embedded microorganisms (Dang and Lovell, 2015; Flemming and Wingender, 2010). The community composition and productivity of the microorganisms are continuously controlled by various reciprocal interactions between chemical, biological and physical (e.g., hydrological) factors (Gerbersdorf and Wieprecht, 2015).
The operation of reservoirs can lead to significant accumulations of fine sediment and nutrients, which can lead to formation and development of biofilms on the sediment surface. Microorganisms and their metabolic activities support fundamental ecological and biochemical processes, e.g. biodegradation of organic matter and toxins, water self-purification and the cycling of nutrients (Battin et al., 2016; Madsen, 2011; Nicolella et al., 2005; Schultz and Urban, 2008; Shannon et al., 2008), and the community composition eventually determines their ecological and environmental functions (Danovaro and Pusceddu, 2007; Foshtomi et al., 2015; Gilbertson et al., 2012). The microorganisms settled on fine sediments are also known to glue sediment grains together through EPS matrix (Jones, 2017; Paterson et al., 2018) and permeate their void space, which can, in turn, alter sediment properties, e.g. density, morphology, size gradation, architecture (Fang et al., 2012; Gibbs, 1983; Huiming et al., 2011; Shang et al., 2014), erosion and transport behavior of sediments (Banasiak et al., 2005; Chen et al., 2017; Droppo et al., 2015; Fang et al., 2016, 2015, 2017; Gerbersdorf et al., 2008; Malarkey et al., 2015; Righetti and Lucarelli, 2010; Vignaga et al., 2013) and associated contaminants (Burns and Ryder, 2001; Förstner et al., 2004). The ability of biofilms to increase erosion thresholds by biological actions is named “biostabilization” (de Brouwer et al., 2005; Passarelli et al., 2014; Paterson et al., 2018) and has been reported to mediate sediment erosion, transportation, deposition and consolidation (ETDC) cycle in aquatic ecosystems, including reservoirs (Fang et al., 2015). Although the importance of biostabilization at the sediment surface and deeper layers has been increasingly recognized through laboratory and field studies (Chen et al., 2017; Paterson et al., 2018 and references therein) over the past two decades, yet little is known about the transport processes of biofilm-bound sediment and its effect on aquatic environment and bed morphology. Thus, sediment transport models as well as river and reservoir management strategies disregard the effect of biofilm growth on sediment. 
\section{Cyanobacteria}
\label{sec:1}
Reservoirs and dams used for hydropower or drinking water supply are subject to major anthropogenic and environmental influences, potentially favouring harmful cyanobacterial blooms (HCB). Pelagic and benthic cyanobacteria are known to produce a large variety of toxins (e.g. microcystins, saxitoxins, anatoxins), frequently resulting in human and animal poisoning events (Dietrich et al., 2008). There is consensus that increased nutrient input (eutrophication) in conjunction with increasing surface water temperatures promoting water column stratification, are closely linked to more frequent and pronounced HCBs (O’Neil et al., 2012; Paerl and Huisman, 2008; Quiblier et al., 2013). Nutrients that have been mainly linked to the occurrence of eutrophication associated HCBs are phosphorus (P) and nitrogen (N). However, past decades of research revealed that blooms are not restricted to eutrophic or hypertrophic systems, but also occur in nutrient poorer (oligo-mesotrophic) lakes and reservoirs and can even be promoted by oligotrophication (Callieri et al., 2014; Ernst et al., 2009; Posch et al., 2012; Salmaso, 2010; Salmaso et al., 2012).
Which species form blooms?
It is essential to realise that cyanobacteria are not a homogenous group when trying to understand HCB dynamics and bloom response to mitigating or promoting factors and management strategies. Therefore, the first step of any bloom management should be a careful analysis of the in situ situation, including identification of the predominant bloom species. HCB forming cyanobacteria can be clustered into two groups based on their nutrient preferences and their likelihood to form blooms: Cyanobacteria genera typically forming surface blooms in N-limited systems are N2-fixing taxa (Dolichospermum, Cylindrospermopsis, Aphanizomenon, Lyngbya and Nodularia) as they can compensate for N limitation by using atmospheric N2. Planktothrix blooms are typically found in mesotrophic lakes (Ernst et al., 2009), where they produce deep chlorophyll maxima. Genera typically occurring in eutrophic, P- and N-rich systems are Microcystis and Cylindrospermopsis (Dolman et al., 2012; Paerl and Otten, 2013).
Management tools
Several studies have summarised HCB monitoring and management strategies (Chorus, 2012; Chorus and Bartram, 1999; Ibelings et al., 2015; Paerl et al., 2016). The suitability and success of the respective strategy might vary and should be considered carefully on a case-by-case basis (O’Neil et al., 2012; Paerl and Otten, 2013). Nutrient control and reduction is still considered to be the most efficient management tool. Here again, a detailed analysis of the in situ situation should be the first step. This includes a characterisation of the reservoir by its trophic level index and the identification of nutrient sources and their pathways. Nutrients can enter the system via point sources like natural and artificial tributaries or water pumping storages (especially in the case of hydropower reservoirs) or can originate from diffuse sources from the catchment area. The German environment agency (Umweltbundesamt (UBA)) developed a decision support system for reservoir managements (https://toxic-cyanobacteria.com/), which summarises background information and questionnaires linked with cyanobacterial blooms and toxins. This tool might deliver an overview of the situation and help to develop a water safety plan.
Besides the described direct control of nutrient input, a reduction might also be achieved via sediment dredging/flushing and binding or precipitation of nutrients via flocculating agents,  these invasive strategies should however be considered carefully, as they often come with financial and ecological costs (Bullerjahn et al., 2016; Paerl and Otten, 2013). Finally, physical factors are often manipulated to mitigate HCBs. Physical measures include disintegration of vertical stratification and increased flushing rates in reservoirs (Paerl et al., 2016). Vertical mixing devices to break down stratification have been successfully applied in small lakes and reservoirs (Paerl and Otten, 2013). The efficacy and applicability of these physical manipulation primarily depends on the size and purpose of the reservoir, the predominant bloom species and should be applied in parallel with nutrient control (Paerl et al., 2016).
\section{Greenhouse gas emissions}
\label{sec:1}
Dammed systems release substantial amounts of greenhouse gases (GHG) that contribute considerably to the global budget of GHG (Barros et al., 2011; Beaulieu et al., 2014; Hertwich, 2013; Luyssaert et al., 2012; Raymond et al., 2013). Especially CH4 is of major concern with reservoirs emitting 8.9-22.2 t/yr CH4 globally (Deemer et al. 2016).
GHG emissions depend on the location and construction of the reservoir and on its management. The conditions in the catchment determine influx of nutrients and the amount of organic carbon that is eventually stored in the sediments. The meteorological conditions affect the strength of stratification, the duration of the season, the intensity of vertical mixing in the water column and of the gas exchange at the water surface. The morphometry of the reservoir together with meteorological conditions and amount of organic material in the sediments determines, whether anoxic conditions develop in the deep water which support accumulation of CH4 and CO2 and eventually lead to GHG emission during fall overturn, or to dam-downstream emissions if the outlet of the reservoir is located in the anoxic layer.
Reservoir management may alter the efficiency of different GHG emission pathways. The choice of the depth and the temporal sequence of water influx and withdrawal can have substantial effects on GHG emissions. Continuous deep-water withdrawal may remove GHG and nutrients released from the sediments to dam-downstream. Nutrient removal may result in reduced within-system production, thus in a smaller accumulation of organic material in the sediments and consequently a reduced GHG production. The removal of cold deep-water reduces stratification, thus supports vertical mixing of e.g. oxygen. In contrast, withdrawal near the water surface may directly remove algae and cyanobacteria from the system and thus reduce organic material, but favours development of strong stratification with reduced mixing. The latter increases the risk of anoxia and GHG accumulation in the deep water.
In case water removal is not continuous but operated as a single major drawdown of the reservoir, the pressure release on the CH4 saturated pore-water causes formation of CH4 bubbles and thus a substantial increase in CH4 emissions via ebullition (Engle and Melack, 2000; Deborde et al., 2010; Varadharajan and Hemond, 2012; Maeck et al., 2014; Harrison et al., 2017). If this drawdown is conducted after the stratification period, it may additionally cause substantial downstream GHG emissions if CO2 and CH4 had accumulated in anoxic waters. In case small drawdowns are conducted at regular time intervals overall CH4 emissions via ebullition may be smaller than for an operation with no withdrawal followed by a single large drawdown per year.
Inflow of water at largest depth may be advantageous for reducing GHG emissions. Because the inflowing water is typically warmer and enriched in oxygen compared to the deep water, stratification is reduced which supports vertical transport by mixing, and the deep water is oxygenated which enhances oxidation of CH4 at the sediment water interface and prevents accumulation of CH4 that could be released during fall overturn.  
Pump-storage operation with turbination and inflow of water in the deep water may be particularly advantageous with respect to GHG emissions, because the ebullition fluxes during regular small drawdowns lead to comparatively small CH4 emissions and the inflow of water in the deep water supports oxygenation of the deep water and oxidation of CH4 at the sediment-water interface, thus preventing accumulation of CH4. Another effect of regular water fluctuations in a pump-storage system is the periodic drying and wetting of near shore sediments which may reduce CH4 fluxes from these sediments.
\section{Societal implications}
\label{sec:1}
The social surrounding is linked in a myriad of ways to most processes within and around the reservoir (Votruba and Broža, 1989; Zarfl et al., 2015; Kirchherr et al., 2016; Kirchherr and Charles, 2016; World Commission on Dams, 2000; Carnea, 1997; Kornijów, 2009; Siegmund-Schultze et al., 2018). The land use within the upstream catchment area influences sediment erosion (Annandale, 1987), nutrient input e.g. from agricultural uses or sewage treatment plants but also hydrological aspects (Zubala, 2009). Downstream riparian may also have demands regarding water quantity and quality (sediments, nutrients, general suspended load). There are many ecological concerns as some reservoirs, due to their history, do not meet minimal discharge requirements or incorporate pass-through installations for aquatic organisms (WFD, n.d.; BMU and UBA, 2016; Aguiar et al., 2016; Dittmann et al., 2009). People living near the reservoir may benefit in various ways: water storage ensures drinking and irrigation water supply, there are recreational opportunities as well as affiliated economic advantages like employment in hydropower enterprises or tourism. But differing interest of stakeholders can lead to conflict e.g. in connection to agricultural uses and nature protection versus recreation and other economic uses, especially energy production (Nguyen et al., 2017). Invasive species, like aquatic birds, might also be an issue (Kleinhenz and Koenig, 2018). Nature protected areas will also limit possible interferences in the hydrological system therefore will be influencing water in- and output.
Occupancy of water reservoirs and their direct and indirect implications
Renewable energy production is naturally fluctuating. If the reservoir is connected to a pump-storage system, it is possible to follow that variable demand structure particularly if hydropower production is the only or the dominant function (Giesecke and Mosonyi, 2009). As hydro electric production will lead to water level changes within the reservoir but also downstream, recreational facilities on the shore, nature protection areas or downstream riparian floodplain will be subject to drastic water level changes thereby impede full flexibility (Hanson et al., 2007). Depending on the overall system, pump storage might introduce water from a different catchment into the reservoir leading to new challenges e.g. in respect to nutrient content (Godde et al., 2015).
If the reservoir is part of a profitable touristic system, these interests may play a major role in the management of the reservoir. Frequent water level fluctuations, water quality issues and health risks (e.g. due to HCB) will be a topic of considerable concern (Daus et al., 2019; Schrenk-Bergt et al., 2004). There may be conflicts regarding accessibility of the reservoir if it is restricted due to hydro-energy installation or nature protection areas. 
In case of drinking water reservoirs, regulations are in place that ensure water quality and govern water use, restrict access and regulate possible emitters of pollutants within the catchment (§§ 51,52 WHG; TrinkwV). Drinking water supply will usually be the predominant purpose and other uses will only be authorised if they do not impair this vital function. If the implementation of drinking water protection area leads to economic losses for other usages (e.g. agriculture, forestry), compensation is required (§ 52 Abs. 5 WHG, § 96 Abs. 2 WHG). Furthermore, water quantity might be an issue considering dry years.
Many reservoirs are part of a decentralised flood protection system. In case of an imminent flood event, the water level will be lowered to provide additional protection volume. Downstream riparians will benefit from a more constant water flow that will usually not exceed safe water levels. However, hydropower companies will not be in favour of discharging large amounts of water at a time when there might not be much revenue. 
\section{Summary}
\label{sec:1}
The management of water reservoirs affects a multitude of interests and aspects in and around the actual water body depending on the main function of the system. Most reservoirs serve multiple purposes and are therefore prone to interest conflicts between those uses. The CHARM research project developed a tool to showcase some important key aspects of management related issues connected to the five topics under consideration, namely: sedimentation, biostabilisation (of fine sediments), cyanobacteria and possible HCB, greenhouse gas emissions from reservoirs mainly CH4 and CO2 and possible social conflicts and trade-offs. The tool provided shows the possible  interrelations of these topics. Furthermore, a cross-impact-matrix (Weimer-Jehle, 2006) can identify certain interrelations and is open to be modified by the individual users of the tool.
 \cite{RefB} and \cite{RefJ}.
\subsection{Subsection title}
\label{sec:2}
as required. Don't forget to give each section
and subsection a unique label (see Sect.~\ref{sec:1}).
\paragraph{Paragraph headings} Use paragraph headings as needed.
\begin{equation}
a^2+b^2=c^2
\end{equation}

% For one-column wide figures use
\begin{figure}
% Use the relevant command to insert your figure file.
% For example, with the graphicx package use
  \includegraphics{example.eps}
% figure caption is below the figure
\caption{Please write your figure caption here}
\label{fig:1}       % Give a unique label
\end{figure}
%
% For two-column wide figures use
\begin{figure*}
% Use the relevant command to insert your figure file.
% For example, with the graphicx package use
  \includegraphics[width=0.75\textwidth]{example.eps}
% figure caption is below the figure
\caption{Please write your figure caption here}
\label{fig:2}       % Give a unique label
\end{figure*}
%
% For tables use
\begin{table}
% table caption is above the table
\caption{Please write your table caption here}
\label{tab:1}       % Give a unique label
% For LaTeX tables use
\begin{tabular}{lll}
\hline\noalign{\smallskip}
first & second & third  \\
\noalign{\smallskip}\hline\noalign{\smallskip}
number & number & number \\
number & number & number \\
\noalign{\smallskip}\hline
\end{tabular}
\end{table}


\begin{acknowledgements}
The authors would like to thank all project members of the CHARM project to faciliate this publication with a lot of effort. Namely this would be Prof. Silke Wiprecht, Dr. Stefan Haun, Prof. Rüdiger Glaser, Dr. Sabine Gerbersdorf, Prof. Frank Peeters, Prof. Daniel Dietrich, Dr. Hilmar Hoffmann and Dr. Dominic Martin-Creutzburg.
\end{acknowledgements}


% Authors must disclose all relationships or interests that 
% could have direct or potential influence or impart bias on 
% the work: 
%
\section*{Conflict of interest}
The authors declare that they have no conflict of interest.



% BibTeX users please use one of
%\bibliographystyle{spbasic}      % basic style, author-year citations
%\bibliographystyle{spmpsci}      % mathematics and physical sciences
%\bibliographystyle{spphys}       % APS-like style for physics
%\bibliography{}   % name your BibTeX data base

\begin{thebibliography}{}
\bibitem{RefJ}
Aguiar, F.C., Martins, M.J., Silva, P.C., Fernandes, M.R., 2016. Riverscapes downstream of hydropower dams. Effects of altered flows and historical land-use change. Landscape and Urban Planning 153, 83-98.
Atkinson, E. 1996. The Feasibility of Flushing Sediment from Reservoirs, TDR Project R5839, Report OD 137, HR Wallingford.
Banasiak, R., Verhoeven, R., De Sutter, R., Tait, S., 2005. The erosion behaviour of biologically active sewer sediment deposits: Observations from a laboratory study. Water Res. 39, 5221–5231. https://doi.org/10.1016/j.watres.2005.10.011
Barros, N., Cole, J.J., Tranvik, L.J., Prairie, Y.T., Bastviken, D., Huszar, V.L.M., del Giorgio, P., Roland, F. (2011). Carbon emission from hydroelectric reservoirs linked to reservoir age and latitude. Nature Geoscience, 4, 593-596.
Battin, T.J., Besemer, K., Bengtsson, M.M., Romani, A.M., Packmann, A.I., 2016. The ecology and biogeochemistry of stream biofilms. Nat. Rev. Microbiol. 14, 251–263. https://doi.org/10.1038/nrmicro.2016.15
Beaulieu, J.J., Smolenski, R.L., Nietch, C.T., Townsend-Small, A., Elovitz, M.S. (2014). High methane emissions from a midlatitude reservoir draining an agricultural watershed. Environmental Science & Technology, 48 (19), 11100-11108. https://doi.org/10.1021/es501871g.
Beckers, F.; Haun, S.; Gerbersdorf, S.U.; Noack, M.; Dietrich, D.; Martin-Creuzburg, D.; Peeters, F.; Hofmann, H.; Glaser, R.; Wieprecht, S. CHARM - CHAllenges of Reservoir Management - Meeting environmental and social requirements. Hydrolink 2018, 16–18.
Berke, A.P., Turner, L., Berg, H.C., Lauga, E., 2008. Hydrodynamic Attraction of Swimming Microorganisms by Surfaces. Phys. Rev. Lett. 101, 038102. https://doi.org/10.1103/PhysRevLett.101.038102
Block, J.C., Haudidier, K., Paquin, J.L., Miazga, J., Levi, Y., 1993. Biofilm accumulation in drinking water distribution systems. Biofouling 6, 333–343. https://doi.org/10.1080/08927019309386235
Breitung, V. 2009. Statement on dredging at the barrage Iffezheim (spillway channel) (Rhine-km 332,800-333,900) (in German). Technical report, Bundesanstalt für Gewässerkunde (BfG).
Bruk, S. 1996. Reservoir sedimentation and sustainable management of water resources – the international perspective, International conference on reservoir sedimentation, Fort Collins, USA, 9th – 13th September 1996.
Bullerjahn, G.S., McKay, R.M., Davis, T.W., Baker, D.B., Boyer, G.L., D’Anglada, L.V., Doucette, G.J., Ho, J.C., Irwin, E.G., Kling, C.L., Kudela, R.M., Kurmayer, R., Michalak, A.M., Ortiz, J.D., Otten, T.G., Paerl, H.W., Qin, B., Sohngen, B.L., Stumpf, R.P., Visser, P.M., Wilhelm, S.W., 2016. Global solutions to regional problems: Collecting global expertise to address the problem of harmful cyanobacterial blooms. A Lake Erie case study. Harmful Algae 54, 223–238. https://doi.org/10.1016/j.hal.2016.01.003
Burns, A., Ryder, D.S., 2001. Potential for biofilms as biological indicators in Australian riverine systems. Ecol. Manag. Restor. 2, 53–64. https://doi.org/10.1046/j.1442-8903.2001.00069.x
Callieri, C., Bertoni, R., Contesini, M., Bertoni, F., 2014. Lake Level Fluctuations Boost Toxic Cyanobacterial “Oligotrophic Blooms.” PLoS ONE 9. https://doi.org/10.1371/journal.pone.0109526
Chen, X.D., Zhang, C.K., Zhou, Z., Gong, Z., Zhou, J.J., Tao, J.F., Paterson, D.M., Feng, Q., 2017. Stabilizing Effects of Bacterial Biofilms: EPS Penetration and Redistribution of Bed Stability Down the Sediment Profile. J. Geophys. Res. Biogeosciences 122, 3113–3125. https://doi.org/10.1002/2017JG004050
Chorus, I., 2012. Current approaches to Cyanotoxin risk assessment, risk management and regulations in different countries 151.
Chorus, I., Bartram, J., 1999. Toxic cyanobacteria in water: a guide to their public health consequences, monitoring and management. xv + 416 pp.
Dang, H., Lovell, C.R., 2015. Microbial Surface Colonization and Biofilm Development in Marine Environments. Microbiol. Mol. Biol. Rev. MMBR 80, 91–138. https://doi.org/10.1128/MMBR.00037-15
Danovaro, R., Pusceddu, A., 2007. Biodiversity and ecosystem functioning in coastal lagoons: Does microbial diversity play any role? Estuar. Coast. Shelf Sci., Biodiversity and Ecosystem Functioning in Coastal and Transitional Waters 75, 4–12. https://doi.org/10.1016/j.ecss.2007.02.030
Daus, M.; Koberger, K.; Gnutzmann, N.; Hertrich, T.; Glaser, R., 2019. Transferring Water While Transforming Landscape: New Societal Implications, Perceptions and Challenges of Management in the Reservoir System Franconian Lake District. Water 2019, 11(12), 2469; https://doi.org/10.3390/w11122469
de Brouwer, J.F.C., Wolfstein, K., Ruddy, G.K., Jones, T.E.R., Stal, L.J., 2005. Biogenic Stabilization of Intertidal Sediments: The Importance of Extracellular Polymeric Substances Produced by Benthic Diatoms. Microb. Ecol. 49, 501–512. https://doi.org/10.1007/s00248-004-0020-z
Deemer, B.R., Harrison, J.A., Li, S., Beaulieu, J.J., DelSontro, T., Barros, N., Bezerra-Neto, J.F., Powers, S.M., dos Santos, M.A., Vonk, J.A., 2016. Greenhouse Gas Emissions from Reservoir Water Surfaces: A New Global Synthesis. BioScience 66, 949–964. https://doi.org/10.1093/biosci/biw117
Deborde, J., Anschutz, P., Guérin, F., Poirier, D., Marty, D., Boucher, G., Thouzeau, G., Canton, M., Abril, G. (2010). Methane sources, sinks and fluxes in a temperate tidal Lagoon: The Arcachon lagoon (SW France). Estuarine, Coastal and Shelf Science, Elsevier, 89, 256-266 (hal-00524642)
DelSontro, T., McGinnis, D.F., Sobek, S., Ostrovsky, I., Wehrli, B. (2010). Extreme methane emissions from a Swiss hydropower reservoir: Contribution from bubbling sediments. Environmental Science & Technology, 44 (7), 2419-2425. https://doi.org/10.1021/es9031369
Dietrich, D.R., Fischer, A., Michel, C., Hoeger, S.J., 2008. Toxin mixture in cyanobacterial blooms – a critical comparison of reality with current procedures employed in human health risk assessment, in: Hudnell, H.K. (Ed.), Cyanobacterial Harmful Algal Blooms: State of the Science and Research Needs, Advances in Experimental Medicine and Biology. Springer New York, pp. 885–912. https://doi.org/10.1007/978-0-387-75865-7_39
Dittmann, R.; Froehlich, F.; Pohl, R.; Ostrowski, M. (2009): Optimum multi-objective reservoir operation with emphasis on flood control and ecology. In: Nat. Hazards Earth Syst. Sci., 9.
Dolman, A.M., Rücker, J., Pick, F.R., Fastner, J., Rohrlack, T., Mischke, U., Wiedner, C., 2012. Cyanobacteria and Cyanotoxins: The Influence of Nitrogen versus Phosphorus. PLOS ONE 7, e38757. https://doi.org/10.1371/journal.pone.0038757.
Döll, P., Fiedler, K., Zhang, J., 2009. Global-scale analysis of river flow alterations due to water withdrawals and reservoirs. Hydrol. Earth Syst. Sci. 13, 2413–2432. https://doi.org/10.5194/hess-13-2413-2009.
DPA. Wasserknappheit: Umweltbundesamt warnt vor Streit ums Trinkwasser , 5 July 2019. Available online: https://www.zeit.de/wirtschaft/2019-07/trinkwasser-knappheit-sommer-trockenheit-umweltbundesamt (accessed on 10 July 2019).
Droppo, I.G., D’Andrea, L., Krishnappan, B.G., Jaskot, C., Trapp, B., Basuvaraj, M., Liss, S.N., 2015. Fine-sediment dynamics: towards an improved understanding of sediment erosion and transport. J. Soils Sediments 15, 467–479. https://doi.org/10.1007/s11368-014-1004-3
Encinas Fernandez, J., Peeters, F., Hofmann, H. (2014). Importance of the autumn overturn and anoxic conditions in the hypolimnion for the annual methane emissions from a temperate lake. Environmental Science & Technology, 48, 7297-7304. https://doi.org/10.1021/es4056164
Engle, D., Melack, J.M. (2000). Methane emissions from an Amazon floodplain lake: Enhanced release during episodic mixing and during falling water. Biogeochemistry, 51 (1), 71-90. https://doi.org/10.1023/A:1006389124823
Ernst, B., Höger, S.J., O´Brien, E., Dietrich, D.R., 2009. Abundance and toxicity of Planktothrix rubescens in the pre-alpine Lake Ammersee, Germany. Harmful Algae 8, 329–342. https://doi.org/10.1016/j.hal.2008.07.006
Europäisches Parlament und Europarat (23.10.2007): Richtlinie 2007/60/EG des Europäischen Parlaments und des Rates vom 23. Oktober 2007 über die Bewertung und das Management von Hochwasserrisiken. EG-HWRM-RL. In: Amtsblatt der Europäischen Union (L 288), S. 27–34.
Fan, J. and Morris, G.L. 1992b. Reservoir Sedimentation. II: Reservoir Desiltation and Long-Term Storage Capacity, Journal of Hydraulic Engineering, ASCE, 118(3).
Fang, H., Fazeli, M., Cheng, W., Dey, S., 2016. Transport of biofilm-coated sediment particles. J. Hydraul. Res. 54, 631–645. https://doi.org/10.1080/00221686.2016.1212938
Fang, H., Fazeli, M., Cheng, W., Huang, L., Hu, H., 2015. Biostabilization and Transport of Cohesive Sediment Deposits in the Three Gorges Reservoir. PLoS ONE 10. https://doi.org/10.1371/journal.pone.0142673
Fang, H., Zhao, H., Shang, Q., Chen, M., 2012. Effect of biofilm on the rheological properties of cohesive sediment. Hydrobiologia 694, 171–181. https://doi.org/10.1007/s10750-012-1140-y
Fang, H.W., Lai, H.J., Cheng, W., Huang, L., He, G.J., 2017. Modeling sediment transport with an integrated view of the biofilm effects. Water Resour. Res. 53, 7536–7557. https://doi.org/10.1002/2017WR020628
Fischer, H., Pusch, M., 2001. Comparison of bacterial production in sediments, epiphyton and the pelagic zone of a lowland river. Freshw. Biol. 46, 1335–1348. https://doi.org/10.1046/j.1365-2427.2001.00753.x
Flemming, H.-C., 2020. Biofouling and me: My Stockholm syndrome with biofilms. Water Res. 173, 115576. https://doi.org/10.1016/j.watres.2020.115576
Flemming, H.-C., Wingender, J., 2010. The biofilm matrix. Nat. Rev. Microbiol. 8, 623–633. https://doi.org/10.1038/nrmicro2415
Flemming, H.-C., Wuertz, S., 2019. Bacteria and archaea on Earth and their abundance in biofilms. Nat. Rev. Microbiol. 17, 247–260. https://doi.org/10.1038/s41579-019-0158-9
Förstner, U., Heise, S., Schwartz, R., Westrich, B., Ahlf, W., 2004. Historical Contaminated Sediments and Soils at the River Basin Scale. J. Soils Sediments 4, 247. https://doi.org/10.1007/BF02991121
Foshtomi, M.Y., Braeckman, U., Derycke, S., Sapp, M., Gansbeke, D.V., Sabbe, K., Willems, A., Vincx, M., Vanaverbeke, J., 2015. The Link between Microbial Diversity and Nitrogen Cycling in Marine Sediments Is Modulated by Macrofaunal Bioturbation. PLOS ONE 10, e0130116. https://doi.org/10.1371/journal.pone.0130116
Geesey, G.G., Mutch, R., Costerton, J.W., Green, R.B., 1978. Sessile bacteria: An important component of the microbial population in small mountain streams 1. Limnol. Oceanogr. 23, 1214–1223. https://doi.org/10.4319/lo.1978.23.6.1214
Gerbersdorf, S.U., Jancke, T., Westrich, B., Paterson, D.M., 2008. Microbial stabilization of riverine sediments by extracellular polymeric substances. Geobiology 6, 57–69. https://doi.org/10.1111/j.1472-4669.2007.00120.x
Gerbersdorf, S.U., Wieprecht, S., 2015. Biostabilization of cohesive sediments: revisiting the role of abiotic conditions, physiology and diversity of microbes, polymeric secretion, and biofilm architecture. Geobiology 13, 68–97. https://doi.org/10.1111/gbi.12115
Gibbs, R.J., 1983. Effect of natural organic coatings on the coagulation of particles. Environ. Sci. Technol. 17, 237–240. https://doi.org/10.1021/es00110a011
Gilbertson, W.W., Solan, M., Prosser, J.I., 2012. Differential effects of microorganism–invertebrate interactions on benthic nitrogen cycling. FEMS Microbiol. Ecol. 82, 11–22. https://doi.org/10.1111/j.1574-6941.2012.01400.x
Godde, D., Engels, K., Schmid, S., Achatz, R., Haupt, O., Beer, C., Höller, S., Jaberg, H., Miller, B., Heigerth, G., Niemann, A., Perau, E., Schreiber, U., Koch, M.K., Schüttrumpf, H., Pummer, E., Günther, M., Plenker, D., 2015. Pumpspeicherkraftwerke. In: Heimerl, S. (Ed.), Wasserkraftprojekte, Band 2. Ausgewählte Beiträge aus der Fachzeitschrift WasserWirtschaft. Springer Vieweg, Wiesbaden, pp. 277-354.
H. Tuson, H., B. Weibel, D., 2013. Bacteria–surface interactions. Soft Matter 9, 4368–4380. https://doi.org/10.1039/C3SM27705D
Haddeland, I., Heinke, J., Biemans, H., Eisner, S., Flörke, M., Hanasaki, N., Konzmann, M., Ludwig, F., Masaki, Y., Schewe, J., Stacke, T., Tessler, Z.D., Wada, Y., Wisser, D., 2014. Global water resources affected by human interventions and climate change. Proc. Natl. Acad. Sci. 111, 3251–3256. https://doi.org/10.1073/pnas.1222475110
Hanson, T.R.; Upton Hatch, L.; Clonts, H.C. (2007): Reservoir water level impacts on recreation, property and non user values. In Journal of the American water ressource association. 38/4. p.1007-1018.
Harb, G. 2013. Numerical Modeling of Sediment Transport Processes in Alpine Reservoirs. Dissertation, Schriftenreihe zur Wasserwirtschaft der Technischen Universität Graz, Band 73.
Harrison, J.A., Deemer, B.R., Birchfield, M.K., O'Malley, M.T. (2017). Reservoir water-level drawdowns accelerate and amplify methane emission. Environmental Science & Technology, 51(3), 1267-1277. https://doi.org/10.1021/acs.est.6b03185
Haun, S., Kjærås, H., Løvfall, S. and Olsen, N.R.B. 2013. Three-dimensional measurements and numerical modelling of suspended sediments in a hydropower reservoir, Journal of Hydrology 479: 180-188.
Hertwich, E.G. (2013). Addressing biogenic greenhouse gas emissions from hydropower in LCA. Environ. Sci. Technol., 47(17): 9604-11. https://doi.org/10.1021/es401820p
Huiming, Z., Hongwei, F., Minghong, C., 2011. Floc architecture of bioflocculation sediment by ESEM and CLSM. Scanning 33, 437–445. https://doi.org/10.1002/sca.20247
Ibelings, B.W., Backer, L.C., Kardinaal, W.E.A., Chorus, I., 2015. Current approaches to cyanotoxin risk assessment and risk management around the globe. Harmful Algae 49, 63–74. https://doi.org/10.1016/j.hal.2014.10.002
Jones, S., 2017. Goo, glue, and grain binding: Importance of biofilms for diagenesis in sandstones. Geology 45, 959–960. https://doi.org/10.1130/focus102017.1
Kirchherr, J., Charles, K.J., 2016. The social impacts of dams. A new framework for scholarly analysis. Environmental Impact Assessment Review 60, 99-114.
Kirchherr, J., Pohlner, H., Charles, K.J., 2016. Cleaning up the big muddy. A meta-synthesis of the research on the social impact of dams. Environmental Impact Assessment Review 60, 115-125.
Kleinhenz, A., Koenig, A., 2018. Home ranges and movements of resident graylag geese (Anser anser) in breeding and winter habitats in Bavaria, South Germany. PloS one 13, e0202443.
Kondolf, G.M., Gao, Y., Annandale, G.W., Morris, G.L., Jiang, E.,  Zhang, J., Cao, Y., Carling, P., Fu, K., Guo, Q., Hotchkiss, R.,  Peteuil, Ch., Sumi, T., Wang, H.-W.,Wang, Z., Wei, Z., Wu, B.,  Wu, C. and  Ted Yang, Ch.T. 2014. Sustainable sediment management in reservoirs and regulated rivers: Experiences from five continents. Earth’s Future 2(5): 256-280.
Kornijów, R., 2009. Controversies around dam reservoirs. Benefits, costs and future. Ecohydrology & Hydrobiology 9, 141-148.
Li, S. Y., Zhang, Q. F., Bush, R. T. and Sullivan, L. A. (2015). Methane and CO2 emissions from China's hydroelectric reservoirs: a new quantitative synthesis. Environmental Science and Pollution Research International 22(7): 5325–5339.
Lozupone, C.A., Knight, R., 2007. Global patterns in bacterial diversity. Proc. Natl. Acad. Sci. 104, 11436. https://doi.org/10.1073/pnas.0611525104
Luyssaert, S., Abri,l G., Andres, R., Bastviken, D., Bellassen, V., Bergamaschi, P., Bousquet, P., Chevallier, F., Ciais, P., Corazza, M., Dechow, R., Erb, K.H., Etiope, G., Fortems-Cheiney, A., Grassi, G., Hartmann, J., Jung, M., Lathière, J., Lohila, A., Mayorga, E., Moosdorf, N., Njakou, D.S., Otto, J., Papale, D., Peters, W., Peylin, P., Raymond, P., Rödenbeck, C., Saarnio, S., Schulze, E.D., Szopa, S., Thompson, R., Verkerk, P.J., Vuichard, N., Wang, R., Wattenbach, M., Zaehle, S. (2012). The European land and inland water CO2, CO, CH4 and N2O balance between 2001 and 2005. Biogeosciences, 9, 3357-3380. https://doi.org/10.5194/bg-9-3357-2012
Madsen, E.L., 2011. Microorganisms and their roles in fundamental biogeochemical cycles. Curr. Opin. Biotechnol., Energy biotechnology – Environmental biotechnology 22, 456–464. https://doi.org/10.1016/j.copbio.2011.01.008
Maeck, A., DelSontro, T., McGinnis, D. F., Fischer, H., Flury, S., Schmidt, M., Fietzek, P. and Lorke, A. (2013). Sediment trapping by dams creates methane emission hot spots. Environmental Science and Technology 47(15): 8130–8137.
Maeck, A., Hofmann, H., Lorke, A. (2014). Pumping methane out of aquatic sediments - ebullition forcing mechanisms in an impounded river. Biogeosciences, 11 (11), 2925-2938. https://doi.org/10.5194/bg-11-2925-2014
Mahmood, K. 1987. Reservoir sedimentation: impact, extent and mitigation. World Bank Technical Paper. 71, Washington D.C..
Malarkey, J., Baas, J.H., Hope, J.A., Aspden, R.J., Parsons, D.R., Peakall, J., Paterson, D.M., Schindler, R.J., Ye, L., Lichtman, I.D., Bass, S.J., Davies, A.G., Manning, A.J., Thorne, P.D., 2015. The pervasive role of biological cohesion in bedform development. Nat. Commun. 6. https://doi.org/10.1038/ncomms7257
Milliman, J.D., Syvitski, J.P.M., 1992. Geomorphic/tectonic control of sediment discharge to the ocean: the importance of small mountainous rivers. Journal of Geology 100, 325– 344.
Mulligan, M., van Soesbergen, A., Sáenz, L., 2020. GOODD, a global dataset of more than 38,000 georeferenced dams. Sci. Data 7, 31. https://doi.org/10.1038/s41597-020-0362-5.
Nguyen, H., Pham, T., Lobry de Bruyn, L., 2017. Impact of Hydroelectric Dam Development and Resettlement on the Natural and Social Capital of Rural Livelihoods in Bo Hon Village in Central Vietnam. Sustainability 9, 1422.
Nicolella, C., Zolezzi, M., Rabino, M., Furfaro, M., Rovatti, M., 2005. Development of particle-based biofilms for degradation of xenobiotic organic compounds. Water Res. 39, 2495–2504. https://doi.org/10.1016/j.watres.2005.04.016
Nilsson, C., Reidy, C.A., Dynesius, M., Revenga, C., 2005. Fragmentation and Flow Regulation of the World’s Large River Systems. Science 308, 405–408. https://doi.org/10.1126/science.1107887
O’Neil, J.M., Davis, T.W., Burford, M.A., Gobler, C.J., 2012. The rise of harmful cyanobacteria blooms: The potential roles of eutrophication and climate change. Harmful Algae, Harmful Algae--The requirement for species-specific information 14, 313–334. https://doi.org/10.1016/j.hal.2011.10.027
Ometto, J. P., Cimbleris, A. C. P., dos Santos, M. A., Rosa, L. P., Abe, D., Tundisi, J. G., Stech, J. L., Barros, N. and Roland, F. (2013). Carbon emission as a function of energy generation in hydroelectric reservoirs in Brazilian dry tropical biome. Energy Policy 58: 109–116.
Paerl, H.W., Gardner, W.S., Havens, K.E., Joyner, A.R., McCarthy, M.J., Newell, S.E., Qin, B., Scott, J.T., 2016. Mitigating cyanobacterial harmful algal blooms in aquatic ecosystems impacted by climate change and anthropogenic nutrients. Harmful Algae, Global Expansion of Harmful Cyanobacterial Blooms: Diversity, ecology, causes, and controls 54, 213–222. https://doi.org/10.1016/j.hal.2015.09.009
Paerl, H.W., Huisman, J., 2008. Blooms Like It Hot. Science 320, 57–58. https://doi.org/10.1126/science.1155398
Paerl, H.W., Otten, T.G., 2013. Harmful Cyanobacterial Blooms: Causes, Consequences, and Controls. Microb. Ecol. 65, 995–1010. https://doi.org/10.1007/s00248-012-0159-y
Passarelli, C., Olivier, F., Paterson, D.M., Meziane, T., Hubas, C., 2014. Organisms as cooperative ecosystem engineers in intertidal flats. J. Sea Res., Trophic significance of microbial biofilm in tidal flats 92, 92–101. https://doi.org/10.1016/j.seares.2013.07.010
Paterson, D.M., Hope, J.A., Kenworthy, J., Biles, C.L., Gerbersdorf, S.U., 2018. Form, function and physics: the ecology of biogenic stabilisation. J. Soils Sediments 18, 3044–3054. https://doi.org/10.1007/s11368-018-2005-4
Pohlert, T., Hillebrand, G., and Breitung, V. 2011. Trends of persistent organic pollutants in the suspended matter of the river rhine. Hydrological Processes, 25:3803–3817.
Posch, T., Köster, O., Salcher, M.M., Pernthaler, J., 2012. Harmful filamentous cyanobacteria favoured by reduced water turnover with lake warming. Nat. Clim. Change 2, 809–813. https://doi.org/10.1038/nclimate1581
Quiblier, C., Susanna, W., Isidora, E.-S., Mark, H., Aurélie, V., Jean-François, H., 2013. A review of current knowledge on toxic benthic freshwater cyanobacteria – Ecology, toxin production and risk management. Water Res. 47, 5464–5479. https://doi.org/10.1016/j.watres.2013.06.042
Raymond, P.A., Hartmann, J., Lauerwald, R., Sobek, S., McDonald, C., Hoover, M., Butman, D., Striegl, R., Mayorga, E., Humborg, C., Kortelainen, P., Dürr, H., Meybeck, M., Ciais, P., Guth, P. (2013). Global carbon dioxide emissions from inland waters. Nature, 503 (7476), 355-9. https://doi.org/10.1038/nature12760
Righetti, M., Lucarelli, C., 2010. Resuspension phenomena of benthic sediments: The role of cohesion and biological adhesion. River Res. Appl. 26, 404–413. https://doi.org/10.1002/rra.1296
Salmaso, N., 2010. Long-term phytoplankton community changes in a deep subalpine lake: responses to nutrient availability and climatic fluctuations. Freshw. Biol. 55, 825–846. https://doi.org/10.1111/j.1365-2427.2009.02325.x
Salmaso, N., Buzzi, F., Garibaldi, L., Morabito, G., Simona, M., 2012. Effects of nutrient availability and temperature on phytoplankton development: a case study from large lakes south of the Alps. Aquat. Sci. 74, 555–570. https://doi.org/10.1007/s00027-012-0248-5
Schrenk-Bergt, C., Krause, D., Lewandowski, J., Steinberg, C.E.W., 2004. Eutrophication Problems and Their Potential Solutions in the Artificial Shallow Lake Altmühlsee (Germany). Studia Quarternaria 21, 73-86.
Schultz, P., Urban, N.R., 2008. Effects of bacterial dynamics on organic matter decomposition and nutrient release from sediments: A modeling study. Ecol. Model. 210, 1–14. https://doi.org/10.1016/j.ecolmodel.2007.06.026
Shang, Q., Fang, H., Zhao, H., He, G., Cui, Z., 2014. Biofilm effects on size gradation, drag coefficient and settling velocity of sediment particles. Int. J. Sediment Res. 29, 471–480. https://doi.org/10.1016/S1001-6279(14)60060-3
Shannon, M.A., Bohn, P.W., Elimelech, M., Georgiadis, J.G., Mariñas, B.J., Mayes, A.M., 2008. Science and technology for water purification in the coming decades. Nature 452, 301–310. https://doi.org/10.1038/nature06599
Shen, H.W. 1999. Flushing sediment through reservoirs, IAHR Journal of Hydraulic Research, 37(6), 743–757.
Siegmund-Schultze, M., do Carmo Sobral, M., Alcoforado de Moraes, M.M.G., Almeida-Cortez, J.S., Azevedo, J.R.G., Candeias, A.L., Cierjacks, A., Gomes, E.T.A., Gunkel, G., Hartje, V., Hattermann, F.F., Kaupenjohann, M., Koch, H., Köppel, J., 2018. The legacy of large dams and their effects on the water-land nexus. Reg Environ Change 18, 1883-1888.
Sobek, S., DelSontro, T., Wongfun, N., Wehrli, B. (2012). Extreme organic carbon burial fuels intense methane bubbling in a temperate reservoir. Geophysical Research Letters, 39 (1). https://doi.org/10.1029/2011gl050144
Stoodley, P., Sauer, K., Davies, D.G., Costerton, J.W., 2002. Biofilms as Complex Differentiated Communities. Annu. Rev. Microbiol. 56, 187–209. https://doi.org/10.1146/annurev.micro.56.012302.160705
Sumi, T. and Kantoush, S.A. 2011.Sediment management strategies for sustainable reservoir. In: Dams and Reservoirs under Changing Challenges, Eds. Schleiss, A.J. and Boes R.M. CRC Press. ISBN 9780415682671.
Syvitski, J.P.M., Vörösmarty, C.J., Kettner, A.J., Green, P., 2005. Impact of Humans on the Flux of Terrestrial Sediment to the Global Coastal Ocean. Science 308, 376–380. https://doi.org/10.1126/science.1109454
Varadharajan, C., Hemond, H.F. (2012). Time-series analysis of high-resolution ebullition fluxes from a stratified, freshwater lake. Journal of Geophysical Research: Biogeosciences, 117 (G2). https://doi.org/10.1029/2011JG001866
Vignaga, E., Sloan, D.M., Luo, X., Haynes, H., Phoenix, V.R., Sloan, W.T., 2013. Erosion of biofilm-bound fluvial sediments. Nat. Geosci. 6, 770–774. https://doi.org/10.1038/ngeo1891
Vörösmarty, C.J., Meybeck, M., Fekete, B. and Sharma, K. 1997. The potential impact of neo-Castorization on sediment transport by the global network of rivers. In: Walling, D.E., and Probst, J.L. (Eds.). Human Impact on Erosion and Sedimentation. (Proc. Rabat Symposium, April 1997) (IAHS Publication 245). Wallingford, UK: 261–273
Vörösmarty, C.J., Meybeck, M., Fekete, B., Sharma, K., Green, P. and Syvitski, J.P.M. 2003. Anthropogenic sediment retention: major global impact from registered river impoundments. Global and Planetary Change, Vol. 39. Amsterdam, Elsevier Science, 169–90.
W.A.: Verordnung über die Qualität von Wasser für den menschlichen Gebrauch (Trinkwasserverordnung) vom 01.01.2003, Neufassung vom 10.03.2016. Online: http://www.landesrecht-bw.de/jportal/?quelle=jlink&query=TrinkwV&psml=bsbawueprod.psml&max=true&aiz=true
Walling, D.E. and Webb, B.W. 1996. Erosion and Sediment Yield: Global and Regional Perspectives, IAHS Publication, 236, 3–19.
Wasserversorgung Kleine Kinzig (WKK) (2019): Geschäftsbericht 2018. Anhaltende Hitzewelle führt zu neuer Rekordabgabe. Online: https://zvwkk.de/content/3-betriebsbesichtigung/modules/4-section-11u2zj8/wkk_geschaeftsbericht_2018.pdf
Weimer-Jehle, W. (2006): Cross-Impact Balances: A System-Theoretical Approach to Cross-Impact Analysis. Technological Forecasting and Social Change, 73:4, 334-361. 
Yoon, Y.N. 1992. The state and the perspective of the direct sediment removal methods from reservoirs, International Journal of Sediment Research, 7(2), 99–116.
Zarfl, C.; Lumsdon, A.E.; Berlekamp, J.; Tydecks, L.; Tockner, K. A global boom in hydropower dam construction. Aquat Sci 2015, 77, 161–170, doi:10.1007/s00027-014-0377-0.
Zubala, T., 2009. Influence of dam reservoir on the water quality in a small upland river. Ecohydrology & Hydrobiology 9, 165-173.

\bibitem{RefB}
Annandale, G.W., 1987. Reservoir sedimentation. Elsevier, Amsterdam.
Batuca, D. and Jordaan, J. 2000. Silting and Desilting of Reservoirs. A.A. Balkema, Rotterdam.
Blackbourn, D.; Rennert, U. Die Eroberung der Natur. Eine Geschichte der deutschen Landschaft, 3rd ed.; Pantheon: München, Germany, 2008; ISBN 9783570550632
Bundesministerium für Umwelt, Naturschutz, Bau und Reaktorsicherheit, Umweltbundesamt, 2016. Integrierte Gewässerbewirtschaftung mit der Wasserrahmenrichtlinie - Deutschlands Gewässer 2015.
Carnea, Michael, M., 1997. Hydropower Dams and Social Impacts. A Sociological Perspective.
Deutscher Bundestag (31.07.2009): Gesetz zur Ordnung des Wasserhaushalts (Wasserhaushaltsgesetz). WHG, legal validity: 04.12.2018), Berlin.
Giesecke, J., Mosonyi, E., 2009. Wasserkraftanlagen. Planung, Bau und Betrieb. Springer, Heidelberg.
Glaser, R. Global change. Das neue Gesicht der Erde; Primus Verlag: Darmstadt, Germany, 2014; ISBN 9783863120993.
International Commission on Large Dams. Talsperren in Deutschland; Springer Vieweg: Wiesbaden, Germany, 2013; ISBN 978-383-48210-7-2.
Milliman, J.D. and Farnsworth, K.L. 2011. River Discharge to the Coastal Ocean - A Global Synthesis. Cambridge  University  Press.
Morris, G.L. and Fan, J. 1998. Reservoir Sediment Handbook. McGraw-Hill Book Co., New York
Votruba, L., Broža, V., 1989. Water management in reservoirs. Elsevier, Amsterdam.
White, R. 2001. Evacuation of sediments from reservoirs. Thomas Telford Publishing.
World Commission on Dams, 2000. Dams and development. A new framework for decision-making. Earthscan, London.

\end{thebibliography}

\end{document}

